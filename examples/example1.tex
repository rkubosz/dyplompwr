\documentclass[12pt,openany,doublesided]{mwrep}
\usepackage[OT4]{polski}

\usepackage[utf8]{inputenc}
\usepackage[a4paper,left=25mm,right=25mm,top=25mm,bottom=25mm]{geometry}
\usepackage{xcolor}
\usepackage[absolute]{textpos}
\usepackage{graphicx}
%\definecolor{jasnoszary}{rgb}{.74 .74 .74}
\usepackage{pwrdtyt}



\usepackage{lipsum}
\usepackage{url}

\author{Arkadiusz Kupczyński}
\title{Wpływ proszków żelaza karbonylkowego i~Terfenolu-D na właściwości magnetomechaniczne kompozytu na bazie żywicy epoksydowej}
\promotor{dr hab. inż. Jerzy Kaleta, prof. nadzw. PWr. dr inż. Daniel Lewandowski}
\wydzial{Wydział Chemiczny}
\kluczowe{słowo0, słowo1, słowo2, słowo3, słowo4,\ldots}
\streszczenie{Krótka praca o niczym}


\begin{document}

\maketitle

\tableofcontents

\listoffigures

\listoftables

%Ala ma kota

\chapter{Ala ma kota}

\lipsum[1]

\section{Ola ma psa}

\lipsum[2-3]

\subsection{Ula ma małpę}

\lipsum[4-10]

\begin{figure}
\includegraphics[width=.4\textwidth]{kotek}
\caption{Ala ma kota}
\end{figure}

\lipsum[11-15]

\begin{table}
\caption{Co kto ma \cite{harel_rzecz_2008}}
\begin{tabular}{|l|l|l|}
\hline
Ala & ma & kota \\
\hline
Ola & ma & psa \\
\hline
Ula & ma & małpę\\
\hline
\end{tabular}
\end{table}

\lipsum[16-20]

\begin{equation}
\sum_{i=1}^{\infty}a_i
\end{equation}

\nocite{*}
\bibliography{biblioex}
\bibliographystyle{plplain}

\end{document}