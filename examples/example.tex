\documentclass[magister,druk,polski]{dyplom}
\usepackage{hyperref}

\usepackage[toc]{appendix}
\renewcommand{\appendixtocname}{Dodatki}
\renewcommand{\appendixpagename}{Dodatki}


\usepackage{listings}
\noappendicestocpagenum
\usepackage{lipsum}
\usepackage{url}


\author{Imię i nazwisko autora.}
\title{Tytuł  polski.}
\titlen{English title.}
\promotor{dr hab. inż. Pan Promotor}
\wydzial{Wydział Chemiczny}
\miejscowosc{Wrocław}
\kluczowe{słowa kluczowe, raz, dwa, trzy}
\streszczenie{Kilka słów o pracy zwane też streszczeniem}
\begin{document}





\maketitle
\documentclass[12pt,a4paper]{article}
\usepackage[utf8]{luainputenc}
\usepackage{lmodern}
\usepackage{polyglossia}
\setdefaultlanguage{polish}
\usepackage{verbatim}
\usepackage{url}
\usepackage[colorlinks=true]{hyperref}
%\usepackage{cleveref}
\author{Robert Kubosz \\ \href{mailto:kubosz.robert@gmail.com}{kubosz.robert@gmail.com}}
\title{Szablon pracy dyplomowej PWr}
\begin{document}
\maketitle
\section{Wstęp}
\label{sec:wstep}
\par Podany szablon pracy dyplomowej powstał na bazie szablonów stworzonych przez dra Wojciecha Myszkę\footnote{\url{https://kmim.wm.pwr.edu.pl/myszka/projekty/klasa-do-skladu-pracy-dyplomowej-magisterskiej-i-inzynierskiej-na-wydziale-mechanicznym-politechniki-wroclawskiej/}}~i~dra Andrzeja Giniewicza\footnote{\url{https://github.com/aginiewicz/pwrmgr}}.
Ja je zmodyfikowałem na potrzeby wymagań Wydziału Chemicznego\footnote{\url{http://www.wch.pwr.edu.pl/druki_dyplomanci,11.dhtml}}. Ten szablon może być użyty przez studentów innych wydziałów.
\par Szablon tworzy stronę tytułową i~dba o właściwy układ dokumentu:
\begin{itemize}
    \item wersja dla archiwum posiada marginesy 2,5 cm dla góry i~dołu strony, 2 cm dla zewnętrznego marginesu oraz 3,5 cm dla wewnętrznego marginesu oraz generowany jest pdf do druku dwustronnego z interlinią 1;
    \item wersja dla promotora i~do ładnego wydruku posiada marginesy 2,5 cm dla góry, dołu i~prawego marginesu, lewy margines ma 3,5 cm, wygenerowany pdf jest do wydruku jednostronnego z interlinią 1,5.
\end{itemize}
Pakiet oferuje możliwość stworzenia angielskiej wersji strony tytułowej zarówno dla pracy magisterskiej jak i~inżynierskiej.

\section{Wymagania}
\par W celu wygenerowania ładnej strony tytułowej i~w~pełni zgodnej  z logotypem PWr\footnote{\url{http://pwr.edu.pl/uczelnia/o-politechnice/materialy-promocyjne/logotyp}} i~wymaganiami Wydziału Chemicznego należy użyć dodatkowych czcionek. Niestety, nie są one dostępne domyślnie w~dystrybucjach \LaTeX, stąd trzeba je zainstalować ręcznie.

\subsection{Instalacja potrzebnych czcionek (systemy uniksopodobne)}
Potrzebne dodatkowe czcionki to Garamond i~Zapf Humanist. W~tym celu można wykorzystać instalator, który zainstaluje odpowiedniki tych fontów: URW Garamond oraz Classico. 
\par Angielska instrukcja do instalacji znajduje się pod linkowanym adresem\footnote{\url{https://www.tug.org/fonts/getnonfreefonts/}}.
\section{Instalacja pakietu}
\subsection{Instalacja w~katalogu domowym użytkownika (systemy uniksopodobne)}
Katalog z pakietem można umieścić w~katalogu:
\begin{verbatim}
    \home\TWOJA_NAZWA_UŻYTKOWNIKA\texmf\tex\latex
\end{verbatim}
Utwórz ten katalog, jeśli go nie masz.

\section{Zastosowanie}
\subsection{Ładowanie pakietu}
\par Pakiet należy załadować w~pliku .tex w~następujący sposób:\\
\verb|\documentclass[opcje]{dyplom}| gdzie za \verb|opcje| można wstawić:
\begin{itemize}
    \item typ pracy: \verb|magister| lub \verb|inzynier| wygenerują strony tytułowe z odpowiednimi nagłówkiem;
    \item formatowanie: \verb|archiwum| lub \verb|druk| nadają pracy dyplomowej odpowiednie formatowanie -- patrz sekcja \ref{sec:wstep};
    \item język strony tytułowej: \verb|polski| lub \verb|english| zmieniają logo PWr i~napisy na stronie tytułowej.
\end{itemize}
Jeśli się nie poda opcji przy ładowaniu klasy, to domyślnie jest ustawiany szablon pracy magisterskiej w~formatowaniu do archiwum i~ze stroną tytułową polską.
Można też te opcje podać i~wtedy wygląda to tak:
\begin{verbatim}
    \documentclass[magister,archiwum,polski]{dyplom}
\end{verbatim}.
Gdy chcemy mieć np. angielską wersję strony tytułowej pracy inżynierskiej i~w~formatowaniu 'do druku', wtedy pakiet należy tak załadować:
\begin{verbatim}
    \documentclass[inzynier,druk,english]{dyplom}
\end{verbatim}.
\subsection{Metadane}
Strona tytułowa pracy dyplomowej zawiera takie dane, jak tytuł polski i~angielski pracy, nazwisko autora, nazwisko promotora, nazwę wydziału, miejscowość, słowa kluczowe i~streszczenie. W~tym celu należy je podać w~formie jak na przykładzie poniżej:
\begin{verbatim}
    \documentclass[magister,druk,polski]{dyplom}
    \author{imię i nazwisko autora pracy}
    \title{tytuł polski}
    \titlen{tytuł angielski}
    \promotor{tytuł, imię i nazwisko promotora}
    \wydzial{nazwa wydziału}
    \miejscowosc{Wrocław lub inna miejscowość}
    \kluczowe{słowa kluczowe}
    \streszczenie{Tutaj możesz napisać swoje krótkie streszczenie 
        pracy dyplomowej.}
\end{verbatim}
W~celu wprowadzenia swoich danych należy zamienić w~powyższym przykładzie odpowiednie frazy na swoje dane, np.:
\begin{verbatim}
    \promotor{tytuł, imię i nazwisko promotora}
\end{verbatim}
na
\begin{verbatim}
    \promotor{dr inż. Jan Kowalski}
\end{verbatim}
Ważne jest, by te słowa kluczowe znajdowały się w~pliku .tex przed 
\verb|\begin{document}|.
\end{document}
\tableofcontents
\listoffigures
\listoftables

%Ala ma kota


\chapter{Ala ma kota}

ĄĆĘŁŃÓŚŹŻ

ąćęłńóśźż

\lipsum[1]

\section{Ola ma psa}

\lipsum[2-3]

\section{Zosia ma kosia}

\lipsum[7]

\subsection{Ula ma małpę}

\lipsum[4-10]

\begin{figure}
\includegraphics[width=.4\textwidth]{kotek}
\caption{Ala ma kota}
\end{figure}

\lipsum[11-15]

\begin{table}
\caption{Co kto ma \cite{harel_rzecz_2008} (patrz też dodatek~\ref{Dod1})}
\begin{tabular}{|l|l|l|}
\hline
Ala & ma & kota \\
\hline
Ola & ma & psa \\
\hline
Ula & ma & małpę\\
\hline
\end{tabular}
\end{table}

\lipsum[16-20]

\begin{equation}
\sum_{i=1}^{\infty}a_i
\end{equation}

\chapter{Listingi}

\begin{lstlisting}
int main()
{
   int a=2*3;
   printf("**Ala ma kota\n**");
   while(!I2C_CheckEvent(I2C1, I2C_EVENT_MASTER_MODE_SELECT)); /* EV5 */
   return 0;
}
\end{lstlisting}


\appendixpage
\appendix
\addappheadtotoc

\chapter{To powinien być dodatek}\label{Dod1}

\lipsum[9-11]

\nocite{*}
\bibliography{biblioex}
\bibliographystyle{dyplom}

\end{document}
